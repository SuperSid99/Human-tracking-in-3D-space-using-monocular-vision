\chapter{Introduction}


The main focus of this thesis is tracking people/objects in 3D space using a monocular vision input. Tracking is not a new problem and has been studied in different settings using different approaches. Keeping track of moving objects in our field of vision is something that we as humans — as well as animals—do all the time. My method will help people understand the things going on in the surrounding environment better, to efficiently track moving objects in our field of vision we need a good understanding of the surrounding environment. It is a complex task that requires a lot of calculations and the use of more than just our visual senses. Repeated and continuous uses of our senses make complex neural connections over time, making humans able to recognize and follow objects within view, in essence, tracking them as they move through the field of vision.\newline

Apart from being a necessary skill for living beings, tracking moving objects also has quite a few other applications; for example, self-driving, and autonomous vehicles need to be aware of their surroundings and keep track of where the objects in their surroundings are going to progress safely. Augmented and virtual reality have an essential use case for precise and accurate tracking of people moving around in confined spaces. Security monitoring cameras and video footage can also benefit from having a tracking capability. However, finding the precise location of a person or an object is a task that requires data in different formats, possibly in the form of depth information from multiple camera streams for different views or angles, or in the form of preplaced trackers and an array of sensors that find the position of these said trackers. Through the scope of this thesis, we will come up with a technique and an algorithm that can track moving objects/people with a reduced amount of data streams for more efficiency. The only input we will have for this study will be a monocular vision input stream from a single fixed camera, making tracking more difficult.\newline

Detecting objects/people in image space is not a new task and can be done by a bunch of different methods such as image segmentation, computer vision, neural networks, etc. However, the challenge lies in transforming the position from the 2D image space to a 3D positioning system without having extra depth information. Keeping track of the trajectory of people in the 3D space using monocular vision is rather difficult and mostly untouched. This already difficult task only becomes more complex when the people being tracked are being obstructed by other objects and are only partially visible. Figuring out the depth from monocular video streams requires smart use of the information present in the input streams. All images captured from regular cameras exist in perspective views that have vanishing points in them; using these vanishing points is the key to finding the depth of an object present in the image.\newline

In Chapter 2, we will go over the related work to our problem statement. In Chapter 3 we will formalize the problem statement and discuss the background and formalities needed to properly understand the setup, the experiment, the solution, and the math behind it. In Chapter 4 we will go over our solution to the problem statement. In Chapter 5 we will discuss the results of the research done in this thesis. Chapter 6 will provide a summary of the work done in this thesis and also provide potential future work that can be done to enhance this topic and research.
